\usepackage{xcolor}
\usepackage{forloop}
\usepackage{ifthen}
\usepackage{calc}
\usepackage{xspace}
\usepackage{mathrsfs}

\newcounter{i}
\newcounter{j}
\newcounter{n1}

\newcommand{\concat}{%
  \mathbin{\raisebox{0.5ex}{\scalebox{.7}{$\frown$}}}%
}

\renewcommand{\epsilon}{\varepsilon}

\newcommand{\iso}{\ensuremath{\cong}}
\newcommand{\symdif}{\ensuremath{\bigtriangleup}}
\newcommand{\forces}{\ensuremath{\Vdash}}
\newcommand{\gl}[2]{\ensuremath{\text{GL}\parens{#1, #2}}}

%\newcommand{\reed}[1]{\relax}
%\newcommand{\Fix}[1]{\relax}
\newcommand{\reed}[1]{{\color{magenta}\bfseries [#1]}}
\newcommand{\Fix}[1]{{\color{red}\bfseries [#1]}}
\newcommand{\Comment}[1]{}
\newcommand{\Space}[1]{}
\newcommand{\Num}[1]{#1}

\newcommand{\injects}{\ensuremath{\hookrightarrow}}
\newcommand{\into}{\injects}
\newcommand{\surjects}{\ensuremath{\twoheadrightarrow}}
\newcommand{\onto}{\surjects}

\newcommand{\interior}[1]{\ensuremath{\text{int}\parens{#1}}}

\DeclarePairedDelimiter\ceil{\lceil}{\rceil}
\DeclarePairedDelimiter\floor{\lfloor}{\rfloor}

\newtheorem{theorem}{Theorem}
\newtheorem{lemma}{Lemma}
\newtheorem{proposition}{Proposition}
\newtheorem{claim}{Claim}
\newtheorem{corollary}{Corollary}

\newcommand{\nat}{\ensuremath{\mathbb{N}}}
\newcommand{\real}{\ensuremath{\mathbb{R}}}
\newcommand{\integers}{\ensuremath{\mathbb{Z}}}
\newcommand{\rational}{\ensuremath{\mathbb{Q}}}
\newcommand{\complex}{\ensuremath{\mathbb{C}}}

\newcommand{\nonzero}[1]{\ensuremath{#1_{\neq 0}}}
\newcommand{\greaterset}[2]{\ensuremath{#1_{> #2}}}
\newcommand{\lesserset}[2]{\ensuremath{#1_{< #2}}}
\newcommand{\subgroup}[2]{\ensuremath{#1 < #2}}
\newcommand{\normalsubgroup}[2]{\ensuremath{#1 \triangleleft #2}}
\newcommand{\id}[1]{\ensuremath{\bm{1}_{#1}}}

\newcommand{\notationNonZero}[1][X]{Let $\nonzero{#1}$ denote the set of $x \in #1$ so that $x \neq 0$.\xspace}
\newcommand{\notationGreaterSet}[1][X]{Let $\greaterset{#1}{y}$ denote the set of $x \in #1$ so that $x > y$.\xspace}
\newcommand{\notationLesserSet}[1][X]{Let $\lesserset{#1}{y}$ denote the set of $x \in X$ so that $x < y$.\xspace}
\newcommand{\notationSubgroup}[1][H]{Let $\subgroup{#1}{G}$ denote that $#1$ is a subgroup of $G$.\xspace}
\newcommand{\notationNormalSubgroup}[1][N]{Let $\normalsubgroup{#1}{G}$ denote that $#1$ is a normal subgroup of $G$.\xspace}
\newcommand{\notationIdentity}[1][X]{Let $\id{#1}$ denote the identity function on $#1$.\xspace}

% \Sym{X} is the set of bijections from X to X
\newcommand{\Sym}[1]{\ensuremath{\text{Sym}\parens{#1}}}
\newcommand{\Aut}[1]{\ensuremath{\text{Aut}\parens{#1}}}
\newcommand{\notationSym}[1][X]{Let $\Sym{X}$ be the group of bijections of $X$.\xspace}

\newcommand{\toplim}{\ensuremath{\text{T}\lim}}
\newcommand{\toplimup}{\ensuremath{\overline{\toplim}}}
\newcommand{\toplimlow}{\ensuremath{\underline{\toplim}}}

\newcommand{\emptysingleton}{\ensuremath{\curlys{\emptyset}}}
\newcommand{\withoutempty}{\ensuremath{\setminus \emptysingleton}}

\newcommand{\isnat}[1]{\ensuremath{#1 \in \nat}}
\newcommand{\seq}[3]{\ensuremath{{\left( #1_{#2} \right)}_{#2 \in #3}}}
\newcommand{\setint}[3]{\ensuremath{\bigcap_{#2 \in #3} #1_{#2}}}
\newcommand{\natinti}[2]{\ensuremath{\setint{#1}{#2}{\nat}}}
\newcommand{\natint}[1]{\ensuremath{\natinti{#1}{n}}}
\newcommand{\natseqi}[2]{\ensuremath{\seq{#1}{#2}{\nat}}}
\newcommand{\natseq}[1]{\ensuremath{\natseqi{#1}{n}}}

\newcommand{\diam}[1]{\ensuremath{\text{diam} \left( #1 \right)}}

\newcommand{\eventually}{\ensuremath{\forall^{\infty}}}
\newcommand{\infinitelymany}{\ensuremath{\exists^{\infty}}}

\newcommand{\power}[1]{\ensuremath{\mathcal{P} \left( #1 \right)}}
\newcommand{\card}[1]{\ensuremath{\left| #1 \right|}}

\newcommand{\brackets}[3]{\ensuremath{{\left#1 {#3} \right#2}}}
\newcommand{\parens}[1]{\brackets{(}{)}{#1}}
\newcommand{\angles}[1]{\brackets{<}{>}{#1}}
\newcommand{\curlys}[1]{\brackets{\{}{\}}{#1}}
\newcommand{\squares}[1]{\brackets{[}{]}{#1}}
\newcommand{\parensum}[3]{\ensuremath{\parens{\sum_{#1}^{#2} {#3}}}}

\newcommand{\poly}[2]{\ensuremath{{#1}\left[ #2 \right]}}

\newcommand{\tand}{\ensuremath{~\text{and}~}}
\newcommand{\tor}{\ensuremath{~\text{or}~}}
\newcommand{\tsuchthat}{\ensuremath{~\text{s.t.}~}}

\newcommand{\beatree}[1]{\ensuremath{\subseteq {#1}^{\leq \nat}}}

\newcommand{\metricspace}[2]{\ensuremath{\parens{{#1},{#2}}}}

\newcommand{\setbuild}[2]{\ensuremath{\left\{ {#1} : {#2} \right\}}}

\newcommand{\openball}[3][]{\ensuremath{B_{#1}\parens{{#2},{#3}}}}
\newcommand{\closedball}[3][]{\ensuremath{\overline{B}_{#1}\parens{{#2},{#3}}}}

\newcommand{\powerset}[1]{\ensuremath{\mathcal{P}\parens{#1}}}
\newcommand{\powersetfin}[1]{\ensuremath{\mathcal{P}_{\text{fin}}\parens{#1}}}

\newcommand{\subgroupgen}[1]{\ensuremath{{\left< {#1} \right>}}}

\newcommand{\idmatrix}[1]{%
    \setcounter{n1}{#1 - 1}
    \begin{pmatrix}
        \forloop{i}{0}{\value{i} < #1}{%
            \forloop{j}{0}{\value{j} < #1}{%
                \ifthenelse{\equal{\value{i}}{\value{j}}}{1}{0}
                \ifthenelse{\value{j} < \value{n1}}{&}{}
            }
            \ifthenelse{\value{i} < \value{n1}}{\\}{}
        }
    \end{pmatrix}
}

\newcommand{\twobytwo}[4]{\ensuremath{\begin{pmatrix} #1 & #2 \\ #3 & #4 \end{pmatrix}}}

\newcommand{\after}{\circ}


\newcommand{\finitecommabody}[1]{}
\newcommand{\countablecommabody}[1]{}
\newcommand{\finitecomma}[2]{%
    \renewcommand{\finitecommabody}[1]{#1}%
    \ensuremath{\finitecommabody{0}, \finitecommabody{1}, \ldots, \finitecommabody{#2 - 1}}}
\newcommand{\finiteset}[2]{\ensuremath{\curlys{\finitecomma{#1}{#2}}}}
\newcommand{\countablecomma}[1]{\ensuremath{%
    \renewcommand{\countablecommabody}[1]{#1}%
    \ensuremath{\countablecommabody{0}, \countablecommabody{1}, \ldots}}}
\newcommand{\countableset}[1]{\ensuremath{\curlys{\countablecomma{#1}}}}

\newcommand{\woutlog}{without loss of generality\xspace}
\newcommand{\Woutlog}{Without loss of generality\xspace}

\newcommand{\actOn}[2]{\ensuremath{#1 \cdot #2}}
\newcommand{\kernel}[1]{\ensuremath{\text{ker}\parens{#1}}}

\newcommand{\deffuncparam}[6]{%
    \expandafter\newcommand\csname domain#1\endcsname{\ensuremath{#3}}%
    \expandafter\newcommand\csname codomain#1\endcsname{\ensuremath{#5}}%
    \expandafter\newcommand\csname decl#1\endcsname[1]{\ensuremath{#2_{##1} : #3 #4 #5}}%
    \expandafter\newcommand\csname funcname#1\endcsname[1]{\ensuremath{#2_{##1}}}%
    \expandafter\newcommand\csname body#1\endcsname[2]{\ensuremath{#6}}%
    \expandafter\newcommand\csname call#1\endcsname[2]{\ensuremath{#2_{##1} \parens{##2}}}%
    \expandafter\newcommand\csname callbody#1\endcsname[2]{\ensuremath{\expandafter\csname call#1\endcsname{##1}{##2} = \expandafter\csname body#1\endcsname{##1}{##2}}}%
    \expandafter\newcommand\csname map#1\endcsname[2]{\ensuremath{##2 \mapsto \expandafter\csname body#1\endcsname{##1}{##2} }}%
    \expandafter\newcommand\csname inversecall#1\endcsname[2]{\ensuremath{#2_{##1} \parens{##2}^{-1}}}%
    \expandafter\newcommand\csname image#1\endcsname[1]{\ensuremath{\text{im}\parens{#2_{##1}}}}%
}

\newcommand{\defbinfunc}[6]{%
    \expandafter\newcommand\csname domain#1\endcsname{\ensuremath{#3}}%
    \expandafter\newcommand\csname codomain#1\endcsname{\ensuremath{#5}}%
    \expandafter\newcommand\csname decl#1\endcsname{\ensuremath{#2 : #3 #4 #5}}%
    \expandafter\newcommand\csname funcname#1\endcsname{\ensuremath{#2}}%
    \expandafter\newcommand\csname body#1\endcsname[2]{#6}%
    \expandafter\newcommand\csname call#1\endcsname[2]{\ensuremath{#2 \parens{##1, ##2}}}%
    \expandafter\newcommand\csname callpair#1\endcsname[1]{\ensuremath{#2 \parens{##1}}}%
    \expandafter\newcommand\csname callbody#1\endcsname[2]{\ensuremath{\expandafter\csname call#1\endcsname{##1}{##2} = \expandafter\csname body#1\endcsname{##1}{##2}}}%
    \expandafter\newcommand\csname map#1\endcsname[2]{\ensuremath{\parens{##1, ##2} \mapsto \expandafter\csname body#1\endcsname{##1}{##2} }}%
}

\newcommand{\defbinop}[6]{%
    \expandafter\newcommand\csname domain#1\endcsname{\ensuremath{#3}}%
    \expandafter\newcommand\csname codomain#1\endcsname{\ensuremath{#5}}%
    \expandafter\newcommand\csname decl#1\endcsname{\ensuremath{#2 : #3 #4 #5}}%
    \expandafter\newcommand\csname funcname#1\endcsname{\ensuremath{#2}}%
    \expandafter\newcommand\csname body#1\endcsname[2]{#6}%
    \expandafter\newcommand\csname call#1\endcsname[2]{\ensuremath{##1 #2 ##2}}%
    \expandafter\newcommand\csname callbody#1\endcsname[2]{\ensuremath{\expandafter\csname call#1\endcsname{##1}{##2} = \expandafter\csname body#1\endcsname{##1}{##2}}}%
    \expandafter\newcommand\csname map#1\endcsname[2]{\ensuremath{\parens{##1, ##2} \mapsto \expandafter\csname body#1\endcsname{##1}{##2} }}%
}

\newcommand{\deffunc}[6]{%
    \expandafter\newcommand\csname domain#1\endcsname{\ensuremath{#3}}%
    \expandafter\newcommand\csname codomain#1\endcsname{\ensuremath{#5}}%
    \expandafter\newcommand\csname decl#1\endcsname{\ensuremath{#2 : #3 #4 #5}}%
    \expandafter\newcommand\csname funcname#1\endcsname{\ensuremath{#2}}%
    \expandafter\newcommand\csname body#1\endcsname[1]{#6}%
    \expandafter\newcommand\csname call#1\endcsname[1]{\ensuremath{#2 \parens{##1}}}%
    \expandafter\newcommand\csname callbody#1\endcsname[1]{\ensuremath{\expandafter\csname call#1\endcsname{##1} = \expandafter\csname body#1\endcsname{##1}}}%
    \expandafter\newcommand\csname map#1\endcsname[1]{\ensuremath{##1 \mapsto \expandafter\csname body#1\endcsname{##1} }}%
    \expandafter\newcommand\csname invcall#1\endcsname[1]{\ensuremath{#2^{-1}\parens{##1}}}%
    \expandafter\newcommand\csname image#1\endcsname{\ensuremath{\text{im}\parens{#2}}}%
}

\newcommand{\defproj}[3]{%
    \defbinfunc{#1#2}{\pi_{#2}}{#2 \times #3}{\to}{#2}{##1}%
    \defbinfunc{#1#3}{\pi_{#3}}{#2 \times #3}{\to}{#3}{##2}%
}

% Footnoteable proofs for useful stuff
\newcommand{\notationEventually}{Define the notation $\eventually n, \varphi(n)$ to mean $\exists N \in \nat \tsuchthat \forall n \geq N, \varphi(n)$, where $\varphi$ is some formula.\xspace}
\newcommand{\eventuallyCombinesPrf}{%
    Suppose that $\eventually n, \varphi(n)$ and $\eventually n, \psi(n)$, where $\varphi$ and $\psi$ are two formulas.
    Then there is some $N_1$ so that $\varphi(n)$ holds for all $n \geq N_1$, and some $N_2$ so that $\psi(n)$ holds for all $n \geq N_2$.
    Then define $N = \max(N_1, N_2)$, and suppose $n \geq N$.
    Then $\varphi(n)$ holds because $n \geq N_1$ and $\psi(n)$ holds because $n \geq N_2$, so $\eventually n, \varphi(n) \wedge \psi(n)$.\xspace}


\newcommand{\irrationalDensePrf}{For any rational $q$, take $q + \frac{\sqrt{2}}{n}$, which is irrational. However, because we can make $\frac{1}{n}$ smaller than any $r > 0$, we can make some irrational that will be in $B(q,r)$ for any $r > 0$.\xspace}

